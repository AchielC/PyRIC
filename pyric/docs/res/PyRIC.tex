%%%%%%%%%%%%%%%%%%%%%%%%%%%%%%%%%%%%%%%%%%%%%%%%%%%%%%%%%%%%%%%%%%%%%%%%%%%%%%%
% Copyright (C) 2016  Dale V. Patterson (wraith.wireless@yandex.com)
%
%This program is free software: you can redistribute it and/or modify it under
%the terms of the GNU General Public License as published by the Free Software
%Foundation, either version 3 of the License, or (at your option) any later
%version.
%
%Redistribution and use in source and binary forms, with or without modifications,
%are permitted provided that the following conditions are met:
% o Redistributions of source code must retain the above copyright notice, this
%   list of conditions and the following disclaimer.
% o Redistributions in binary form must reproduce the above copyright notice,
%   this list of conditions and the following disclaimer in the documentation
%   and/or other materials provided with the distribution.
% o Neither the name of the orginal author Dale V. Patterson nor the names of any
%   contributors may be used to endorse or promote products derived from this
%   software without specific prior written permission.
%
% __name__ = 'User Guide'
%__license__ = 'GPL v3.0'
%__version__ = '0.0.2'
%__date__ = 'May 2016'
%__author__ = 'Dale Patterson'
%__maintainer__ = 'Dale Patterson'
%__email__ = 'wraith.wireless@yandex.com'
%__status__ = 'Development'
%%%%%%%%%%%%%%%%%%%%%%%%%%%%%%%%%%%%%%%%%%%%%%%%%%%%%%%%%%%%%%%%%%%%%%%%%%%%%%%

\documentclass[11pt]{article}
\usepackage[utf8]{inputenc}
\setlength{\parindent}{0pt}
\usepackage{graphicx}
\usepackage{listings}
\usepackage{caption}
\usepackage{geometry}
\usepackage{color}
\usepackage[titletoc,toc,title]{appendix}
\usepackage[T1]{fontenc}
 
\definecolor{codegreen}{rgb}{0,0.6,0}
\definecolor{codegray}{rgb}{0.5,0.5,0.5}
\definecolor{codepurple}{rgb}{0.58,0,0.82}
\definecolor{backcolour}{rgb}{0.95,0.95,0.92}

\geometry{left=1.0in,right=1.0in,top=1.0in,bottom=1.0in }

\lstset{
  frame=top,
  frame=bottom,
  basicstyle=\small\normalfont\ ,
  stepnumber=1,
  numbersep=10pt,
  tabsize=2,              
  extendedchars=true,                     
  breaklines=true,
  captionpos=b,
  mathescape=true,
  showspaces=false,           
  showtabs=false,             
  xleftmargin=17pt,
  framexleftmargin=17pt,
  framexrightmargin=17pt,
  framexbottommargin=5pt,
  framextopmargin=5pt,
  showstringspaces=false,
  backgroundcolor=\color{backcolour},   
    commentstyle=\color{codegreen},
    keywordstyle=\color{magenta},
    numberstyle=\tiny\color{codegray},
    stringstyle=\color{codepurple},
    basicstyle=\footnotesize
}

\title{PyRIC v0.0.5: User Manual}
\author{Dale V. Patterson\\ wraith.wireless@yandex.com}

\begin{document}
\maketitle
\tableofcontents

\section{About PyRIC}\label{sec:About}
PyRIC is a python port of (a subset of) iw and by extension a python port of 
Netlink\cite{spae} (w.r.t nl80211 functions). The goal of PyRIc is to provide 
a simple interface to the underlying nl80211 kernel support that handles the 
complex operations of Netlink seamlessy while maintaining a minimum of "code 
walking" to understand, modify and extend. PyRIC will only work on Linux 
with Netlink support.\\

While users can utilize libnl(.py) and libio(.py) to communicate directly with 
the kernel, the true utility of PyRIC is pyw(.py). Like iw, pyw provides an 
interface/buffer between the caller and the kernel, handling all message
construction, parsing and transfer transparently and without requiring any 
Netlink knowledge or experience. \\

At this time, PyRIC (through pyw functions) can:
\begin{itemize}
\item enumerate interfaces and wireless interfaces,
\item get/set hw address,
\item identify a radio's chipset and driver,
\item turn device on/off,
\item get supported standards,
\item get/set regulatory domain,
\item get info on device,
\item add/delete interfaces. 
\end{itemize}
It also provides users with the ability, through libnl(.py) to extend the above 
functionality by creating additional commands.\\

Currently, PyRIC does handle multicast messages i.e. events or dumps although 
plans for future versions include adding the ability to process dumps.

\subsection{Naming Conventions}
The terms interface, device and radio are all used interchangeably throughout to 
refer to a network interface card (NIC) wireless or Ethernet. The following terms 
will always have one one meaning:
\begin{itemize}
\item \textbf{dev} - the device name i.e. wlan0 or eth0 of a NIC, 
\item \textbf{phy} - the physical index of a NIC i.e. the 0 in phy0,
\item \textbf{ifindex} - the interface index of a NIC,
\item \textbf{card} or \textbf{Card} - a NIC abstraction, an object used in pyw 
functions see the following section for a description. 
\end{itemize}

\subsection{Cards}
A Card is merely a wrapper around a tuple t = (phy index,device name,ifindex). 
Since the underlying Netlink calls sometimes require the physical index, sometimes 
the device name, and sometimes the ifindex, pyw functions\footnote{Not all functions 
accept a Card, the function pyw.devinfo() will accept either a Card or a dev and 
pyw.isinterface() only accepts a dev} take a Card object which doesn't require
callers to know which identifier to use for each function. There are four primary methods to creating a Card: 
\begin{enumerate}
\item \textbf{pyw.getcard} returns a Card object from a given dev,
\item \textbf{pyw.devinfo} returns the dict info where info['card'] is the Card 
object. This function will take either a card or a dev
\item \textbf{pyw.devadd} returns a new Card object, 
\item \textbf{pyw.ifaces} returns a list of tuples t = (Card,mode) sharing the 
same phy as a given dev.
\end{enumerate}

A side affect of using Cards is that many of the netlink calls require the ifindex.
The ifindex is found through the use of ioctl, meaning two sockets have to be 
created and two messages have to be sent, received and parsed in order to execute
the command. With Cards, the ifindex is requested for only once. \\

Keep in mind that any identifier (phy,dev,ifindex) can be invalidated outside of 
your control. Another program can rename your interface, that is change the dev 
without your knowledge. Depending on what functions are being used this may not 
be noticed right away as the phy will remain the same. Also, usb devices (if the 
usb is disconnected and reconnected) will have the same dev but the phy and 
ifindex will be different.

\subsection{Hierarchy/Architecture}
PyRIC's hierarchy is briefly discussed next.

\begin{enumerate}
\item \textbf{\_\_init\_\_.py}: Initialize PyRIC, defines the EUNDEF error code 
(PyRIC uses errno for all errorcodes adding EUNDEF) and PyRIC's common exception 
class, 'error' - all submodules use this class for any exceptions.  
\item \textbf{README.md}: self-descriptive
\item \textbf{TODO}: lists any outstanding TODOs, ideas for future revisions
\item \textbf{RFI}: comments and observations about netlinks, nl80211 etc
\item \textbf{channels.py}: ISM and UNII frequencies and channels, with conversion functions
\item \textbf{device.py}: device and chipset utility functions
\item \textbf{pyw.py}: this is the interface, providing wireless interface manipulation functions
\item \textbf{lib}: lib subpackage
\begin{enumerate}
\item \textbf{\_\_init\_\_.py}: initialize lib subpackage
\item \textbf{libnl.py}: netlink API
\item \textbf{libio.py}: ioctl API
\end{enumerate}
\item \textbf{net}: net subpackage
\begin{enumerate}
\item \textbf{\_\_init\_\_.py}: initialize net subpackage
\item \textbf{if\_h.py}: inet/ifreq definition
\item \textbf{sockios\_h.py}: socket-level I/O control call flags
\item \textbf{netlink\_h.py}: port of netlink.h
\item \textbf{genetlink\_h.py}: port of genetlink.h
\item \textbf{policy.py}: defines attribute datatypes
\item \textbf{wireless}: wireless subpackage
\begin{enumerate}
\item \textbf{\_\_init\_\_.py}: initialize lib subpackage
\item \textbf{nl80211\_h.py}: port of nl80211 (commands \& attributes)
\item \textbf{nl80211\_c.py}: nl80211 attribute datatypes/policies
\end{enumerate}
\end{enumerate}
\item \textbf{docs}: docs subpackage
\item \textbf{\_\_init\_\_.py}: initialize docs subpackage
\item \textbf{nlhelp.py}: functions to search display nl80211 constants
\item \textbf{commands.help}: nl80211 commands help data (json)
\item \textbf{attributes.help}: nl80211 attributes help data (json)
\item \textbf{PyRIC.pdf}: this file
\item \textbf{res}:resource subpackage
\begin{enumerate}
\item \textbf{PyRIC.tex}: LaTeX for user guide
\item \textbf{PyRIC.bib}: bibliography for user guide
\end{enumerate}
\end{enumerate}

\section{Installing PyRIC}\label{sec:installing}
PyRIC is self-sufficient in that there are no third-party/external dependencies 
and can be run out of the box. To install, from a terminal type 'pip install 
PyRIC'. This is my first attempt at packaging and it has not been currently 
tested. PyRIC can be downloaded on PyPI at https://pypi.python.org/pypi?name=PyRIC or 
github at https://github.com/wraith-wireless/pyric/releases/latest. This is my first
attempt at packaging and it has not been currently tested. If you decide to download the tarball and run PyRIC outside of it's directory or from another program, you 
should put it on your Python path. Use Google for directions on doing so. 

\section{Using PyRIC}\label{sec:using}
As stated previously, PyRIC provides a set of functions to interact with your 
system's radio(s) and the ability to interact directly with the kernel through
netlink and ioctl sockets. \\

It is expected that the reader has a basic knowledge of netlinks. For a 
review see Graf\cite{libnl}. 

\subsection{Interacting with the Wireless Core and Wireless NICs: pyw.py}
If you can use iw, you can use pyw. The easist way to explain how to use pyw is 
with an example. Imagine your wireless network on ch 6 has been experiencing 
difficulites lately and you want to capture some traffic to analyse it. Listing 
\ref{lst:pentest} shows how to set up a wireless pentest environment. \\

\begin{lstlisting}[caption={Setting up a Wireless Pentest Environment},
                   label={lst:pentest},
                   language=Python]
 1: import pyric                     # pyric error (and ecode EUNDEF)
 2: from pyric import pyw            # for iw functionality
 3: from pyric import device         # for chipset/driver
 4: from pyric.channels import rf2ch # rf to channel conversion
 5:
 6: # list interfaces, wireless interfaces
 7: print "Interfaces: ", pyw.interfaces()
 8: print "Wireless Interfaces: ", pyw.winterfaces()
 9:
10: # get a Card & info for alfa0 & print a description
11: info = pyw.devinfo('alfa0') 
12: card = info['card']
13: driver = device.ifdriver(card.dev)
14: chipset = device.ifchipset(driver)
15: msg = "{Using {0} in mode: {1}".format(card,info['mode'])
16: if info['mode'] == 'managed':
17:    msg += " currently on channel {0} width {1}".format(rf2ch(info['RF']),
18:                                                        info['CHW'])
19: msg += " driver: {0}, chipset: {1}".format(driver,chipset)
20: print msg
21:
22: # prepare a virtual interface named pent0 in monitor mode
23: pdev = 'pent0'
24: pcard = pyw.devadd(card,pdev,'monitor')
25: for iface in pyw.ifaces(card):
26:    if iface[0].dev != pdev:
27:        msg = "deleting {0} in mode {1}".format(iface[0],iface[1])
28:        print msg
29:        pyw.devdel(iface[0])
30: pyw.macset(pcard,'00:03:93:57:54:46')
31: pyw.up(pcard)
32: pyw.chset(pcard,6,None)
33: msg = "Virtual interface {0} in monitor mode on ch 6".format(pcard)
34: print msg + ", using hwaddr: {0}".format(pyw.macget(pcard)
35: 
36: # DO stuff here
37:
38: # restore original
39: print "deleting ", pcard
40: pyw.devdel(pcard)
41: card = pyw.devadd(card,card.dev,info['mode'])
42: pyw.up(card)
43: print "card ", card, " restored"   
\end{lstlisting}

Listing \ref{lst:pentest} attempts to show most of the available pyw functions 
in use and is the basic shell used in another project, Wraith\cite{wraith}, to 
instantiate a wireless (802.11) sensor. Lines 1 and 2 should always be included 
as they import the pyric error and pyw functions. Lines 3 and 4 import ifchipset, 
ifdriver and rf2ch functions. In lines 10 through 20, a Card object for 'alfa0' 
is created and details about the interface are printed. On line 24, a device named 
'pent0' is created in monitor mode and in lines 25 through 29, all interfaces on 
the same phy as the new device are deleted (we have found that it is better to 
delete all interfaces on the same phy ensuring that processes like NetworkManager 
for example, don't interfere with the new device). Then, the mac address is spoofed 
and the device is set to listen on channel 6 NOHT. Starting on line 40, the system 
is restored by deleting the monitor card and restoring the original device, 'alfa0'. 
Note, only 'alfa0' is restored, and not any devices that may have been deleted on 
line 29. Additionally, bringing the device up (line 43) will usually result in the 
device reconnecting although in some situations, a manual reconnect to the AP is 
required. Future versions of pyw will include "connect" functionality to rectify 
this. 

\subsubsection{One-time vs Persistent Sockets}
The example Listing \ref{lst:pentest} uses one-time sockets (netlink and ioctl).
When using iw, there are several things that occur prior to the actual command
or request being submitted. First, iw creates a netlink socket. Then, iw will 
request the family id for nl80211. The relative time spent doing this is neglible 
but, it is redundant and it may become noticeable in programs that repeatedly 
use the Netlink service. Once comlete, iw closes the socket. PyRIC eliminates 
these redundancies by using a global variable in pyw that stores the family id 
after the first time it is requested and by providing callers the option to use 
persistent sockets. 
\begin{itemize}
\item \textbf{One-time Sockets} Similar to iw. The command, creates the netlink 
socket (or ioctl socket), composes the message, sends the message and receives 
the response, parses the results, closes the socket and returns the results to 
the caller. At no time does the caller need to be aware of any underlying Netlink 
processes or structures.
\item \textbf{Persistent Sockets} Communication and parsing only. The onus of 
socket creation and deletion is on the caller which allows them to create one 
(or more) socket(s). The pyw functions will only handle message construction, 
message sending and receiving and message parsing.
\end{itemize}
The caller needs to be cognizant of whether the function requires a netlink or
ioctl socket. Passing the wrong type will result in an error. \\

NOTE: One must remember that there is an upper limit to the number of open netlink
sockets. It is advised to use one-time functions as much as possible and save the
use of persistent sockets for use in code that repeatedly makes use of netlink. \\

The latest version of pyw (v 0.1.*) implements this functionality through the 
use of what I call templates\footnote{I use templates and stubs for the lack
of any better naming convention}, Listing \ref{lst:template} and stubs Listing 
\ref{lst:stub}.

\begin{lstlisting}[caption={A Basic Netlink Function Template},
                   label={lst:template},
                   language=Python]
def fcttemplate(arg0,arg1,..,argn,*argv):
    # put parameter validation (if any) here
    try:
        nlsock = argv[0]
    except IndexError:
        return _nlstub_(fcttemplate,arg0,arg1,...,argn)
        
    # command execution
    ...
    return results
\end{lstlisting}

The template function in Listing \ref{lst:template} checks if argv has a netlink
socket\footnote{ioctl calls operate in the same manner} at index 0. If so, it
proceeds to execution. If there is no socket, the stub is executed which creates 
one. If something other than a netlink socket is at argv[0], an error will be 
raised during execution. \\

\begin{lstlisting}[caption={Function \_nlstub\_},
                   label={lst:stub},
                   language=Python]
def _nlstub_(fct,*argv):
    nlsock = None
    try:
        nlsock = nlsock = nl.nl_socket_alloc()
        argv = list(argv) + [nlsock]
        return fct(*argv)
    except pyric.error: 
        raise # catch & release
    finally:
        if nlsock: nl.nl_socket_free(nlsock)
\end{lstlisting}

The stub function, Listing \ref{lst:stub} allocates a netlink socket, executes
the original (now with a netlink socket) and then destroys the netlink socket.\\

\begin{lstlisting}[caption={Using Persistent Sockets},
                   label={lst:persistent},
                   language=Python]
 1: import pyric                      # pyric error (and ecode EUNDEF)
 2: from pyric import pyw             # for iw functionality
 3: from pyric import device          # for chipset/driver
 4: from pyric.channels import rf2ch  # rf to channel conversion
 5: from pyric.lib import libnl as nl # for netlink sockets
 6: from pyric.lib import libio as io # for ioctl sockets 
 7:
 8: # create our sockets 
 9: nlsock = nl.nl_socket_alloc(timeout=1)
10: iosock = io.io_socket_alloc()
11:
12: # list interfaces, wireless interfaces
13: print "Interfaces: ", pyw.interfaces()
14: print "Wireless Interfaces: ", pyw.winterfaces(iosock):
15:
16: # get a Card & info for alfa0 & print a description
17: info = pyw.devinfo('alfa0',nlsock) 
18: card = info['card']
19: driver = device.ifdriver(card.dev)
20: chipset = device.ifchipset(driver)
21: msg = "{Using {0} in mode: {1}".format(card,info['mode'])
22: if info['mode'] == 'managed':
23:    msg += " currently on channel {0} width {1}".format(rf2ch(info['RF']),
24:                                                        info['CHW'])
25: msg += " driver: {0}, chipset: {1}".format(driver,chipset)
26: print msg
27:
28: # prepare a virtual interface named pent0 in monitor mode
29: pdev = 'pent0'
30: pcard = pyw.devadd(card,pdev,'monitor',nlsock)
31: for iface in pyw.ifaces(card,nlsock):
32:    if iface[0].dev != pdev:
33:        msg = "deleting {0} in mode {1}".format(iface[0],iface[1])
34:        print msg
35:        pyw.devdel(iface[0],nlsock)
36: pyw.macset(pcard,'00:03:93:57:54:46',iosock)
37: pyw.up(pcard,ioctl)
38: pyw.chset(pcard,6,None)
39: msg = "Virtual interface {0} in monitor mode on ch 6".format(pcard)
40: msg += ", using hwaddr: {0}".format(pyw.macget(pcard,iosock)
41: 
42: # DO stuff here
43:
44: # restore original
45: print "deleting ", pcard
46: pyw.devdel(pcard,nlsock)
47: card = pyw.devadd(card,card.dev,info['mode'],nlsock)
48: pyw.up(card,ioctl)
49: print "card ", card, " restored"   
50:
51: # release the sockets
52: nl.nl_socket_free(nlsock)
53: io.io_socket_alloc(iosock)
\end{lstlisting}

In Listing \ref{lst:persistent}, the wireless pentesting environment from Listing 
\ref{lst:pentest} is repeated, this time using persistent sockets. Note the imports 
at lines 5 and 6 for libnl and libio, socket creation at lines 9 and 10 and socket
destructions at lines 52 and 53. Also, note the lack of a socket used in the 
pyw.interfaces() call. This is due to the fact the function interfaces() reads
from the file system and does not use netlink or ioctl. \\

Use Python's built in help features on pyw functions or see Appendex \ref{sec:pywapi} 
to determine what type of 
socket is needed.

\subsection{Interacting with the Kernel: libnl.py and libio.py}
The kernel interfaces, libnl.py and libio.py are located in the lib directory.
They handle socket creation/deletion, message creation/parsing and kernel 
communication. Aside from creating and deleting persistent sockets, there is 
little need to access their functions unless you plan on extending pyw 
functionality. As such, a further discussion of libnl.py and libio.py can be in
the next section.   

\section{Extending PyRIC}\label{sec:extending}
You may find that pyw does not offer some of the functionality you need. Using
libnl.py and/or libnl.io, additional functionality can be added to your program. 

\subsection{Porting C}
All Python ports of C header files can be found in the net directory. C Enums and 
\#defines are ported using constants. C structs are ported using three Python 
structures and the Python struct package:
\begin{enumerate}
\item a format string for packing and unpacking the struct
\item a constant specifying the size of the struct in bytes
\item a function taking the attributes of the struct as arguments and returning
a packed string
\end{enumerate}
Listing \ref{lst:cstruct} shows the C definition of the nlmsghdr found in netlink.h.

\begin{lstlisting}[caption={C Struct nlmsghdr},
                   label={lst:cstruct},
                   language=C]
struct nlmsghdr {
    __u32 nlmsg_len;
    __u16 nlmsg_type;
    __u16 nlmsg_flags;
    __u32 nlmsg_seq;
    __u32 nlmsg_pid;
};
\end{lstlisting}

And Listing \ref{lst:pstruct} shows the ported version in Python.

\begin{lstlisting}[caption={Corresponding Python Definition},
                   label={lst:pstruct},
                   language=Python]
nl_nlmsghdr = "IHHII"
NLMSGHDRLEN = struct.calcsize(nl_nlmsghdr)
def nlmsghdr(mlen,nltype,flags,seq,pid):
    return struct.pack(nl_nlmsghdr,NLMSGHDRLEN+mlen,nltype,flags,seq,pid)
\end{lstlisting}

When using pyw, dealing with these structures is handled transparently by libnl.py 
and libio.py. When extending or customizing pyw, a basic understanding of the 
definitions in netlink\_h.py, genetlink\_h.py and if\_h.py.

\subsection{Input/Output Control (ioctl)}
PyRIC provides more than just iw-related functions, it also implements functions
from ifconfig and iwconfig. These command line tools still use ioctl (or the proc
directory). For example, interfaces() reads from '/proc/net/dev' to retrieve all
system interfaces and winterfaces() use ioctl to check if a device is wireless.
Input/Output control calls have only been used when there was no viable alternative 
and, it should not be necessary to have to add any further ioctl commands. If you
find that you need an ioctl related command, search through if\_h.py for the 
appropriate structure and add it's definitions to ifreq. 

\subsection{Netlink and nl80211}
Documentation on Netlink, and nl80211 in particular, is so minimal as to be 
neglible. The clusterfuck of code and lack of comments in the iw source tree 
make it impossible to use as any sort of roadmap. Fortunately Thomas Graf's 
site\cite{libnl} has excellent coverage of the libnl, the Netlink library.
Using this as a reference, a simple Netlink parser was put together. With 
strace and the parser, Netlink messages could be dissected and analyzed. \\

\begin{appendices}
\section{API: pyw.py}\label{sec:pywapi}

\subsection{Constants}
\begin{itemize}
\item \textbf{\_FAM80211ID\_}: Global netlink family id of nl80211. Do not touch
\item \textbf{IFTYPES}: redefined (from nl80211\_h.py) interface modes
\item \textbf{MNTRFLAGS}: redefined (from nl80211\_h.py) monitor mode flags
\end{itemize}

\subsection{Objects/Classes}
\textbf{Card} A wrapper around a tuple 
\texttt{t = (physical index,device name,interface index)}
which exposes the following properties through '.':
\begin{itemize}
\item \textbf{phy}: physical index
\item \textbf{dev}: device name
\item \textbf{idx}: interface index (ifindex)
\end{itemize}
Because the underlying Netlink calls will sometimes require the physical index,
sometimes the device name, and sometimes the ifindex, pyw functions accept a Card,
object. This allows callers to use pyw functions without having to remember which 
identifier the function requires. However, in some cases the function requires 
a dev or accepts both. See the next section on functions.\\

While callers could create their own Cards, it is recommend to use one of the following
\begin{itemize}
\item \textbf{pyw.getcard} returns a Card object from a given dev
\item \textbf{pyw.devinfo} returns the dict info where info['card'] is the Card 
object. This function will take either a card or a dev
\item \textbf{pyw.devadd} returns a new Card object
\item \textbf{pyw.ifaces} returns a list of tuples t = (Card,mode) sharing the 
same phy as a given dev
to do so. It is also recommended to periodically validate the Card. On some 
cheaper usb wireless nics, there are periodic disconnects which results in a new 
phy and ifindex.
\end{itemize}

\subsection{Functions}
\begin{itemize}
\item interfaces(): (ifconfig), type: filesystem, returns list of all network dev
\item isinterface(dev): (ifconfig <dev>) type: filesystem, check dev is an interface
\item winterfaces([iosock]): (iwconfig), type: ioctl, list wireless interfaces
\item iswireless(dev,[iosock]): (iwconfig <dev>), type: ioctl, check dev is a 
wireless interface
\item regget([nlsock]: (iw reg get), type: netlink, get regulatory domain
\item regset(rd,[nlsock]): (iw reg set <rd>), type: netlink, set regulatory domain 
to rd
\item getcard(dev,[nlsock]) (N/A), type: hybrid netlink and ioctl: get a Card 
object for dev
\item macget(card,[iosock]): (ifconfig card.<dev>), type: ioctl get card's hw
address
\item macset(card,mac,[iosock]): (ifconfig card.<dev> hw ether <mac>), type: 
ioctl, set card's hw address to mac
\item txget(card,[iosock]): (iwconfig card.<dev> | grep Tx-Power card), type: 
ioctl, get card's transmission power
\item up(card,[iosock]) (ifconfig card.<dev> up), type: ioctl, bring card up
\item down(card,[iosock]): (ifconfig card.<dev> down), type: ioctl, bring card 
down
\item devstds(card,[iosock]): (iwconfig card.<dev> | grep IEEE), type: ioctl, get
list of card's 802.11 supported standards
\item validcard(card,[nlsock]): (N/A), type: (hyrbrid netlink and ioctl), verify
card is still valid
\item devinfo(card,[nlsock]): (iw dev card.<dev> info), type: netlink, get info 
for dev 
\item phyinfo(card,[nlsock]): (iw phy card.<phy> info), type: netlink, get info
for phy
\item ifaces(card,[nlsock]): (APX iw card.dev | grep phy\#), type: netlink, get all
cards (w/ modes) of interfaces sharing the same phy as card
\item chget(card,[nlsock]): (iw dev <card.dev> info | grep channelS), type: 
netlink, get card's current channel (only works for cards in mode managed)
\item chset(card,ch,chw,[nlsock]): iw phy <card.phy> set channel <ch> <chw>), 
type: netlink, set card's current channel to ch with width chw
\item devmodes(card,[iosock]): (iw phy card.<phy>), type: netlink, get modes 
supported by card
\item devadd(card,vnic,mode,[flags],[nlsock]): (iw phy card.<phy> interface add
<vnic> type <mode> flags <flags>), type: netlink, creates a new virtual interface
on card's phy with dev vdev, in mode and using flags. Note: flags are only supported
in when creating a monitor mode
\item devdel(card,[nlsock]): (iw card.<dev> del), type: netlink, deletes dev
\begin{itemize}
\item \_hex2mac\_(v): returns a ':' separated mac address from byte stream v
\item \_issetf\_(flags,flag): determines if flag is set in flags
\item \_setf\_(flags,flag): set flag in flags to on
\item \_unsetf\_(flags,flag): set flag in flags to off
\item \_familyid\_(nlsock): returns and sets the Netlink family id for nl80211, 
only called once per module import
\item \_ifindex\_(dev,[iosock]): returns dev's ifindex
\item \_flagsget\_(dev,[iosock]): get's the dev's interface flags
\item \_flagsset\_(dev,flags,[iosock]): set's the dev's interface flags
\item \_iostub\_(fct,*argv): ioctl stub function, calls fct with parameter list argv
and an allocated ioctl socket
\item \_nlstub\_(fct,*argv): netlink stub function, calls fct with parameter list
argv and an allocated netlink socket
\end{itemize}
\end{itemize}

\section{API: libnl.py}\label{sec:libnlapi}
Providing libnl similar functionality, libnl.py provides the interface between
pyw and the underlying nl80211 core. It relates similarily to libnl by providing
functions handling netlink messages and sockets and where possible uses similarly
named functions as those libnl to ease any transitions from C to PyRIC. However,
several liberties have been taken as libnl.py handles only nl80211 generic netlink
messages.

\subsection{Constants}
\begin{itemize}
\item \textbf{BUFSZ} default rx and tx buffer size
\end{itemize}

\subsection{Classes/Objects}
The two classes in libnl.py, NLSocket and GENLMsg, discussed in the following 
sections subclass Python's builtin dict. This has been done IOT to take advantage 
of dict's already existing functions and primarily their mutability and Python's
'pass by name' i.e. modifications in a function will be reflected in the caller.
This makes the classes very similar to the use C pointers to structs in libnl.

\subsubsection{NLSocket}
NLSocket is a wrapper around a netlink socket which exposes the following 
properties through '.':
\begin{itemize}
\item \textbf{sock}: the actual socket
\item \textbf{fd}: the socket's file descriptor (deprecated)
\item \textbf{tx}: size of the send buffer
\item \textbf{rx}: size of the receive buffer
\item \textbf{pid}: port id
\item \textbf{grpm}: group mask 
\item \textbf{seq}: sequence number
\item \textbf{timeout}: socket timeout
\end{itemize}
and has the following methods:
\begin{itemize}
\item incr(): increment sequence number
\item send(pkt): sends pkt returning bytes sent
\item recv(): returns received message (will block unless timeout is set)
\item close(): close the socket
\end{itemize}

NLSockets are created with nl\_socket\_alloc and must be freed with nl\_socket\_free.
See Section \ref{sec:libnlfct}.

\subsubsection{GENLMsg}
GENLMsg is a wrapper around a dict with the following key->value pairs:
\begin{itemize}
\item \textbf{len}: total message length including the header
\item \textbf{nltype}: netlink type
\item \textbf{flags}: message flags
\item \textbf{seq}: seq. \#
\item \textbf{pid}: port id
\item \textbf{cmd}: generic netlink command
\item \textbf{attrs}: list of message attributes. Each attribute is a tuple t =
(attribute,value,datatype) where:
\begin{itemize}
\item \textbf{attribute}: netlink attribute type i.e. CTRL\_ATTR\_FAMILY\_ID
\item \textbf{value}: the unpacked attribute value
\item \textbf{datatype}: datatype of the attribute as defined in nelink\_h i.e.
NLA\_U8
\end{itemize}
\end{itemize}
NOTE: as discussed below, on sending, the seq. \# and port id are overridden with
values of the netlink socket.\\

GENLMsg exposes the following properties:
\begin{itemize}
\item \textbf{len}: length of the message (get only)
\item \textbf{vers}: returns 1 (default version) (get only)
\item \textbf{nltype}: message content i.e. generic or nl80211 (get or set)
\item \textbf{flags}: message flags (get or set)
\item \textbf{seq}: current sequence \# (get or set)
\item \textbf{pid}: port id (get or set)
\item \textbf{cmd}: netlink command (get or set)
\item \textbf{attrs}: attribute list (get only)
\item \textbf{numattrs}: number of attributes (get only)
\end{itemize}

GENLMsg has the following methods:
\begin{itemize}
\item \_\_repr\_\_(): returns a string representation useful for debugging
\item tostream(): returns a packed netlink message
\item nla\_put(v,a,t): appends the attribute a, with value v and datatype t to
the attribute list
\item nla\_put\_<DATATYPE>(v,a): eight specialize functions that append attribute
a with value v and type <DATATYPE> to the attribute list
\item nla\_putat(i,v,a,d): puts attribute a, with value v and datatype d at
index i in the attribute list.
\item nla\_pop(i): removes the attribute tuple at index i, returning the popped
tuple
\item nla\_find(a,value=True): returns the first attribute a. If value returns only
the value otherwise returns the attribute tuple
\item nla\_get(i,value=True): returns the attribute at index i. If value returns only
the value otherwise returns the attribute tuple
\item \_attrpack(a,v,d): (private) packs the attribute tuple 
\end{itemize}

There are two methods of creating a GENLMsg. Create a new message (to send) with nlmsg\_new and create a message from a received packet with nlmsg\_fromstream. These
are discussed below.

\subsection{Functions}\label{sec:libnlfct}
\begin{itemize}
\item \textbf{Netlink Socket Related}
\begin{itemize}
\item nl\_socket\_alloc(pid,grps,seq,rx,tx,timeout): creates a netlink socket 
with port id = pid, group mask = grps, initial seq. \# = seq, send and receive
buffer size = tx and rx respectively and blocking timeout = timeout
\item nl\_socket\_free(sock): closes the socket
\item nl\_socket\_pid(sock): (deprecated for NLSocket.pid) returns the port id
\item nl\_socket\_grpmask(sock): (deprecated for NLSocket.grpmask) returns the 
group mask
\item nl\_sendmsg(sock,msg,override=False): sends the netlink msg over socket. 
NOTE: NLSockets will automatically set the port id and seq. \# regardless of 
their value in the message. If override is True, the message's pid and seq. \# 
will be used instead. 
\item nl\_recvmsg(sock): returns a GENLMsg or blocks unless the socket's timeout 
is set. Should only be called once per every nl\_sendmsg.
\end{itemize}
\item \textbf{Netlink Message Related}
\begin{itemize}
\item nlmsg\_new(nltype=None,cmd=None,pid=None,flags=None,attrs=None): creates a 
new GENLMsg with zero or more attributes defined. 
\item nlmsg\_fromstream(stream): parses the message in stream returning the
corresponding GENLMsg
\item nla\_parse(msg,l,mtype,stream,idx): parses the attributes in stream appending
them to the attribute list of message where msg = the GENLMsg, l = the total length
of the message, mtype = the message content (i.e. netlink type) stream is the
original byte stream and idx is the index of the start of the attribute list
\item nla\_parse\_nested(nested): returns the list of packed nested attributes 
extracted from the stream nested. Callers must unpack and parse the returned 
attributes themselves
\item \_nla\_strip(v): (private) strips padding bytes from the end of v
\item \_maxbufsz\_(): returns the maximum allowable socket buffer size
\end{itemize}
\end{itemize}

\section{Copyright and License}\label{sec:copy}
PYRIC: Python Radio Interface Controller v0.0.5\\

Copyright (C) 2016  Dale V. Patterson (wraith.wireless@yandex.com)\\

This program is free software: you can redistribute it and/or modify it under
the terms of the GNU General Public License\cite{gplv3} as published by the Free
Software Foundation, either version 3 of the License, or (at your option) any 
later version.\\

Redistribution and use in source and binary forms, with or without modifications,
are permitted provided that the following conditions are met:
\begin{itemize}
\item Redistributions of source code must retain the above copyright notice, this
list of conditions and the following disclaimer.
\item Redistributions in binary form must reproduce the above copyright notice,
this list of conditions and the following disclaimer in the documentation and/or 
other materials provided with the distribution.
\item Neither the name of the orginal author Dale V. Patterson nor the names of 
any contributors may be used to endorse or promote products derived from this 
software without specific prior written permission.
\end{itemize}

THIS SOFTWARE IS PROVIDED BY THE COPYRIGHT HOLDER AND CONTRIBUTORS "AS IS" AND
ANY EXPRESS OR IMPLIED WARRANTIES, INCLUDING, BUT NOT LIMITED TO, THE IMPLIED
WARRANTIES OF MERCHANTABILITY AND FITNESS FOR A PARTICULAR PURPOSE ARE DISCLAIMED.
IN NO EVENT SHALL BE LIABLE FOR ANY DIRECT, INDIRECT, INCIDENTAL, SPECIAL,
EXEMPLARY, OR CONSEQUENTIAL DAMAGES (INCLUDING, BUT NOT LIMITED TO, PROCUREMENT
OF SUBSTITUTE GOODS OR SERVICES; LOSS OF USE, DATA, OR PROFITS; OR BUSINESS
INTERRUPTION) HOWEVER CAUSED AND ON ANY THEORY OF LIABILITY, WHETHER IN CONTRACT,
STRICT LIABILITY, OR TORT (INCLUDING NEGLIGENCE OR OTHERWISE) ARISING IN ANY WAY
OUT OF THE USE OF THIS SOFTWARE, EVEN IF ADVISED OF THE POSSIBILITY OF SUCH DAMAGE.\\

PyRIC is free software but use, duplication or disclosure by the United States
Government is subject to the restrictions set forth in DFARS 252.227-7014.\\

Use of this software is governed by all applicable federal, state and local
laws of the United States and subject to the laws of the country where you reside.
The copyright owner and contributors will be not be held liable for use of this
software in furtherance of or with intent to commit any fraudulent or other illegal
activities, or otherwise in violation of any applicable law, regulation or legal
agreement.\\

See http://www.gnu.org/licenses/licenses.html for a copy of the GNU General Public 
License.
\end{appendices}

\bibliographystyle{acm}
\bibliography{PyRIC}
%\addcontentsline{toc}{chapter}{Bibliography}

\end{document}